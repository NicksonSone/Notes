\documentclass{book}
\usepackage{geometry}
\usepackage{fancyhdr}
\pagestyle{fancy}
\cfoot{\thepage}
\usepackage{amsmath, amsthm, amssymb}
\usepackage{graphicx}
\usepackage{hyperref}

\title{Probability Notes}
\author{Nickson Zhu  \\  \url{nicksonsone@gmail.com}}

\begin{document}
\maketitle
\tableofcontents
\newpage

\chapter{Introduction to Probability}
	\section{Interpretation of Probability}
		\subsection{The Frequency Interpretation of Probability}
			\subsubsection{Definition}
			The probablity that some specific outcome of a process can be intrepreted to 
			mean the relative frequency with which the outcome can be obtained if the 
			process is repeated for a large number of times under similar conditions.  

			\subsubsection{Example}
			Toss coin for 1,000,000 times, number of heads is nearly 500,000, but may 
			not exactly 500,000.

			\subsubsection{Shortcoming} 
			\begin{itemize}
				\item number of tests: how large is enough 
				\item similar conditions: conditions cannot be completely the same, otherwise always same outcome
				\item frequency of outcomes: should approximate theoritical probablity, but no permissible variation
				\item repetition: many important problems have no repetition. For instance, probablity of a aquaintance 
			\end{itemize}

		\subsection{Classical Interpretation \& Subjective Interpretation}
			\subsubsection{Classical}
			Based on equally likely outcome.  Paradox: this concept is based on the 
			probablity we are trying to define.  Example: six-sided dice, equally 1/6.

			\subsubsection{Subjective}
			Based on personal belief and information.  

	\section{Experiments and Events}
		\subsection{Definition}
			\subsubsection{Experiment}
			any process in which the possible outcomes can be identified.  
			\subsubsection{Event}
			a well-define set of possible outcomes of the experiment.  

		\subsection{Explanation}
		Not every set of possible outcomes will be called an event.  The probability 
		of an event will be how likely it is that the outcome is in the event.  

	\section{Set Theory}
		\subsection{Definiton}
			\subsubsection{Set}
			Collection of process outcomes of an experiment.  
			\subsubsection{Empty Set}
			Some events are impossible.  
		
			\subsubsection{Infinite set}
			Infinitely many outcomes.  If countable, there 
			is one-to-one correspondence.  If either finite or countable, 
			a set has at most coutably many items.  

		\subsection{Operations on Sets}
			\subsubsection{Union of Sets}
			If $A_1$, $A_2$, \ldots  are countable collection of events,then 
			$\cup_{i=1}^{\infty}$ is also an event.  But $\cup_{i=1}^{\infty}$ 
			does not necessarily have to be an event.  






\begin{thebibliography}{9}
	\bibitem{ConcreteMath}
		Ronald L.   Granham, Donald E.  Knuth, and Oren Patashnik,
		\textit{Concrete Mathematics},
		Addison-Wesley, Reading, MA, 1995.
	\end{thebibliography}
\end{document}
