\documentclass{book}
\usepackage{geometry}
\usepackage{fancyhdr}
\pagestyle{fancy}
\cfoot{\thepage}
\usepackage{amsmath, amsthm, amssymb}
\usepackage{graphicx}
\usepackage{hyperref}

\title{Test document }
\author{Nickson Zhu  \\  \url{nicksonsone@gmail.com}}

\begin{document}
\maketitle
\tableofcontents
\newpage

\chapter{Introduction to Probability}
\section{Interpretation of Probability}
	\subsection{The Frequency Interpretation of Probability}

The probablity that some specific outcome of a process can be intrepreted to 
mean the relative frequency with which the outcome can be obtained if the 
process is repeated for a large number of times under similar conditions.  

\subsubsection{Example}
Toss coin for 1,000,000 times, number of heads is nearly 500,000, but may 
not exactly 500,000.

\subsubsection{Shortcoming} 
\begin{itemize}
\item number of tests: how large is enough 
\item similar conditions: conditions cannot be completely the same, otherwise always same outcome
\item frequency of outcomes: should approximate theoritical probablity, but no permissible variation
\item repetition: many important problems have no repetition. For instance, probablity of a aquaintance 
\end{itemize}

\subsection{Classical Interpretation}
\newpage

\subsection{Another subsection of the firs section}

\section{The second section}


\begin{thebibliography}{9}
	\bibitem{ConcreteMath}
		Ronald L.   Granham, Donald E.  Knuth, and Oren Patashnik,
		\textit{Concrete Mathematics},
		Addison-Wesley, Reading, MA, 1995.
	\end{thebibliography}
\end{document}
